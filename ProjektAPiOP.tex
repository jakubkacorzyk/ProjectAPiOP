\documentclass[11pt]{article}
\setlength\headheight{13.6pt}%
\usepackage[utf8]{inputenc} % Required for inputting international characters
\usepackage[T1]{fontenc} % Output font encoding for international characters
\usepackage{polski}
\usepackage{mathpazo} % Palatino font
\usepackage{graphicx}
\usepackage{fancyhdr}
\usepackage{etoolbox}
\usepackage{blindtext}
\usepackage{geometry}
\geometry{legalpaper, margin=1in}
\usepackage[cleardoublepage=plain]{scrextend}
\graphicspath{ {images/.\ProjectAPiOP} } %change to yours path
\pagestyle{fancy}
\fancyhf{}
\rhead{2017/2018}
\lhead{Projekt z Aspektów Prawnych i Organizacji Przedsiębiorstwa}
\rfoot{Strona \thepage}
\begin{document}
	
	\begin{titlepage} 
	
		\newcommand{\HRule}{\rule{\linewidth}{0.5mm}} % Defines a new command 
		
		\center % Centre everything on the page
		
		%------------------------------------------------
		%	Nagłówki
		%------------------------------------------------
		
		\textsc{\LARGE Akademia Górniczo-Hutnicza im. Stanisława Staszica w Krakowie}\\[1.5cm] % Main heading such as the name of your university/college
		
		\textsc{\Large Projekt z Aspektów Prawnych i Organizacji Przedsiębiorstwa}\\[0.5cm] % Major heading such as course name
		
		\textsc{\large Informatyka EAIiB 2017/2018}\\[0.5cm] % Minor heading such as course title
		
		%------------------------------------------------
		%	Tytuł
		%------------------------------------------------
		
		\HRule\\[0.4cm]
		
		{\huge\bfseries SOFTOPOL - spółka akcyjna}\\[0.4cm] % Title of your document
		
		\HRule\\[1.5cm]
		
		%------------------------------------------------
		%	Authorzy
		%------------------------------------------------
		
		\begin{minipage}{0.4\textwidth}
			\begin{flushleft}
				\large
				\textit{Autorzy}\\
				Jakub \textsc{Kacorzyk} \\
				Bartłomiej \textsc{Łazarczyk} \\
				Szymon \textsc{Jakóbczyk} \\
				Cezary \textsc{Krawczyk} 
			\end{flushleft}
		\end{minipage}
		~
		\begin{minipage}{0.4\textwidth}
			\begin{flushright}
				\large
				\textit{Prowadzący}\\
				dr inż. Marta \textsc{Kraszewska} % Supervisor's name
			\end{flushright}
		\end{minipage}
		
		%------------------------------------------------
		%	Logo
		%------------------------------------------------
		
		\vfill\vfill
		\includegraphics[scale=1.0]{logo.jpg}
		
		%------------------------------------------------
		%	Data
		%------------------------------------------------
		
		\vfill\vfill\vfill % Position the date 3/4 down the remaining page
		
		{\large\today} 
		
		%----------------------------------------------------------------------------------------
		
		\vfill % Push the date up 1/4 of the remaining page
		
	\end{titlepage}
	
	\tableofcontents
	\cleardoublepage
	\setcounter{page}{2}
	
	
	\section{Informacje wstępne}
	
	\subsection{Cel i okoliczności powstania firmy}
		Firma SOFTOPOL została założona przez czterech kolegów ze studiów informatycznych, którzy wspólnie postanowili wykorzystać swoje umiejętności w produktywny sposób.
		
	\subsection{Działalność przedsiębiorstwa}
	
		SOFTOPOL działa w branży gier dostarczając gry z sektora gier mobilnych. Mimo tego swoje produkty kieruje w stronę gracza core'owego (mężczyzna, lat 20 - 35). Wydaje je na platformach internetowych obejmujących obszar Europy, Ameryki Północnej i Australii.
	
	\subsection{Forma prawna przedsiębiorstwa}
	
	Spółka akcyjna jest spółką kapitałową przewidzianą przez kodeks spółek handlowych. Zbiera ona kapitał od akcjonariuszy, czyli udziałowców.
Kapitał spółki akcyjnej podzielony jest na akcje, jeśli SA została dopuszczona do obrotu publicznego, akcje te można nabywać na giełdzie papierów wartościowych.

	\subsection{Postępowanie przygotowawcze}
	
	\section{Rejestracja firmy}
	\subsection{Akt założycielski czyli Statut Spółki}
	
	
Pierwszym krokiem do założenia spółki akcyjnej jest sporządzenie statutu spółki.
	Jest to dokument o charakterze aktu notarialnego, a zatem potrzebować będziemy do tego pomocy notariusza. 
Statut nowo powstałej spółki akcyjnej określa:
\begin{itemize}
\item siedzibę spółki
\item przedmiot działania spółki
\item czas trwania spółki (jeśli takowy jest określony)
\item wysokość kapitału zakładowego, a także kwoty wpłacone przed zarejestrowaniem spółki w ramach pokrycia kapitału zakładowego
\item wartość nominalną akcji oraz liczbę początkową akcji wraz z doprecyzowaniem typu akcji: imienne lub na okaziciela
\item liczbę akcji poszczególnych rodzajów i jakie dają uprawnienia (o ile chcemy to wprowadzić)
\item personalia (nazwisko i imię) lub nazwy założycieli
\item składy liczbowe organów spółki (lub przynajmniej zakres w jakim mają się znajdować):
\begin{itemize}
\item liczbę członków zarządu
\item liczbę członków rady nadzorczej
\end{itemize}
\item podmiot uprawniony do określania składu ww. organów
\item pismo do ogłoszeń (jeśli planujemy dokonywać ogłoszeń)
\item charakterystykę (liczbę i rodzaje) tytułów uczestnictwa w zyskach lub podziale majątku spółki
\item obowiązki wobec spółki wynikające z posiadanych akcji (oprócz konieczności wpłacenia należności za akcje)
\item warunki  i sposoby umorzenia akcji
\item uprawnienia osobiste przyznane akcjonariuszom
\item przybliżoną wartość wszystkich kosztów poniesionych lub obciążających spółkę w związku z jej utworzeniem
\item może również określać inne postanowienia, ale tylko dopuszczane przez Kodeks Spółek Handlowych, w tym m.in.: przyczyny rozwiązania spółki.
\end{itemize}


Statut zaczyna obowiązywać w momencie zawiązania spółki, czyli w chwili objęcia wszystkich akcji.
Osoby, które podpisują Statut są jednocześnie założycielami spółki, której dotyczy.
Zmian w Statucie może po zawiązaniu spółki dokonać tylko Walne Zgromadzenie swoja uchwałą, która zostanie przyjęta kwalifikowaną większością $\frac{3}{4}$ głosów (chyba, że Statut określa bardziej rygorystyczne warunki). Tak podjęta zmiana musi zostać zgłoszona w rejestrze pod rygorem nieważności.

\subsection{Wniesienie kapitału zakładowego}


	Kolejny etapem zakładania spółki akcyjnej jest wniesienie kapitału zakładowego przez akcjonariuszy. Jego minimalna wysokość według Kodeksu Spółek Handlowych to 100 000 zł.
	
	Po dokonaniu wpłat jest on dzielony na akcje o równej wartości nominalnej. Akcje obejmowane dzięki wkładom pieniężnym powinny być opłacone przed zarejestrowaniem spółki co najmniej w $\frac{1}{4}$ ich wartości. Z kolei te obejmowane za wkłady niepieniężne muszą być spłacone najpóźniej w ciągu roku od rejestracji spółki.
	
	Wpłacony kapitał zakładowy jest niepodzielny – nie może zostać rozdzielony między akcjonariuszy, spółka musi utrzymać go w całości. Nie powinien być również wykorzystywany do zaspokojenia roszczeń ubezpieczeniowych.
	
	\subsection{Powołanie organów spółki}
	Kodeks Spółek Handlowych zobowiązuje nowo powstałą spółkę do powołania następujących organów:
	
	\begin{itemize}
	
	\item zarząd spółki, który prowadzi sprawy spółki oraz reprezentuje ją. Jest powoływany na maksymalnie 5 lat. Musi składać się z co najmniej 1 osoby.
	\item radę nadzorczą, która sprawuje (jak sama nazwa wskazuje) nadzór nad wszystkimi działaniami spółki w sposób stały – monitoruje każdą działalność. Składa się z minimum 3 członków (a 5 w spółkach publicznych). Jest powoływana przez walne zgromadzenie.
	\item walne zgromadzenie, które jest najwyższą władzą spółki, a do jego uprawnień należą:
	\begin{itemize}

	\item zatwierdzanie sprawozdań zarządu
	\item udzielanie absolutorium członkom spółki
	\item decydowanie o zbyciu lub wydzierżawieniu przedsiębiorstwa
	\item decydowanie o udziałach w nieruchomościach
	\item emitowanie obligacji
	\item dokonywanie zmian w statucie spółki
	
	\end{itemize}
	
	W posiedzeniu walnego zgromadzenia, zwoływanego przez radę nadzorczą, mogą uczestniczyć posiadacze akcji imiennych, akcji na okaziciela oraz członkowie zarządu i rady nadzorczej.
	
	\end{itemize}
	
	\subsection{Wpis do rejestru sądowego}
	
	Obowiązek dokonania wpisu do Krajowego Rejestru Sądowego ciąży na zarządzie spółki. Do KRS jest zgłaszany jest fakt zawiązania się spółki w formie wniosku podpisanego przez wszystkich członków zarządu. Należy to zrobić najpóźniej 6 miesięcy po sporządzeniu statutu spółki. W przeciwnym razie należy zwrócić akcjonariuszom ich udziały.
	
Do zgłoszenia spółki do sądu rejestrowego wypełniamy formularz KRS-W4 i powinno ono zawierać:
	\begin{itemize}
	
	\item nazwę, siedzibę i adres spółki (lub adres korespondencyjny)
	\item przedmiot działalności spółki zawarty w załączniku KRS-WM
	\item wysokość kapitału zakładowego wraz z liczba i wartością nominalną akcji
	\item w przypadku jeśli została przewidziana w statucie– wysokość kapitału docelowego
	\item rodzaje przywilejów wynikających z posiadanych akcji oraz liczbę akcji uprzywilejowanych
	\item informację o tym, jaka część kapitału zakładowego została pokryta przed rejestracją 
	\item personalia członków zarządu oraz sposób reprezentacji zawarte w załączniku KRS-WK
	\item personalia członków rady nadzorczej
	\item okoliczności wniesienia wkładów niepieniężnych przez wspólników
	\item czas trwania spółki (jeśli jest przewidziany)
	\item oznaczenie pisma do ogłoszeń spółki (jeśli statut je przewiduje)
	\item informację jakie uprawnienia osobiste zostają przyznane określonym akcjonariuszom oraz o tytułach uczestnictwa w dochodach i majątku spółki niewynikającego z akcji

	\end{itemize}
	
	Oprócz wypełnionego wniosku w KRS należy również złożyć:
	
	\begin{itemize}
	
	\item statut spółki
	\item akty notarialne o zawiązaniu spółki i objęciu akcji
	\item oświadczenie wszystkich członków zarządu, mówiące o tym, że wymagane statutem wpłaty na akcje oraz wkłady niepieniężne zostały dokonane zgodnie z prawem
	\item potwierdzony przez bank lub dom maklerski dowód wpłaty na akcje, dokonanej na rachunek spółki w organizacji
	\item dokument stwierdzający ustanowienie organów spółki z wyszczególnieniem ich składu osobowego
	\item zezwolenie lub dowód zatwierdzenia statutu przez właściwy organ władzy publicznej (jeśli są one wymagane do powstania spółki)

	\end{itemize}
	
	\subsection{Zgłoszenie spółki w urzędzie statystycznym,  urzędzie skarbowym i ZUS.}
	
	Dzięki zmianom w przepisach, które nastąpiły 1 grudnia 2014 uzyskanie numerów REGON i NIP (a zatem rejestracja odpowiednio w GUS i US) dzieje się za sprawą sądu rejestrowego. Po złożeniu wniosku o wpis do KRS przesyła on nasze dane do GUS i US, gdzie po około 2 dni od rejestracji w KRS następuje rejestracja naszej spółki. Po uzyskaniu numeru REGON będziemy mogli założyć rachunek bankowy naszej spółki. Sąd rejestrowy przekazuje również nasze podstawowe dane do ZUS.
	
	\subsection{Założenie rachunku bankowego spółki}
	
	Kolejnym etapem zakładania spółki akcyjnej jest stworzenie rachunku bankowego dla spółki. Jest on niezbędny do wpłacenia przez akcjonariuszy swoich udziałów.
	
	  
Wybór banku oraz oferty należy bezpośrednio do samych zainteresowanych (zależnie od aktualnej oferty oraz potrzeb spółki). Do założenia rachunku bankowego dla nowopowstałej spółki będziemy potrzebować:
 \begin{itemize}
 	
	\item numeru REGON uzyskane po dokonaniu zgłoszenia spółki w Urzędzie Statystycznym
	\item akt zawiązania spółki
	\item aktualny odpis KRS 	
 	
 \end{itemize}

	\subsection{Uzupełnienie zgłoszenia w urzędzie skarbowym}


To jednak nie koniec przygody z Urzędem Skarbowym. Po założeniu odpowiedniego konta jesteśmy zobowiązani, w przeciągu 21 dni od wpisu do KRS, złożyć w US formularz NIP-8. Zawiera on podstawowe informacje o spółce oraz dane dotyczące naszych rachunków bankowych. Urząd Skarbowy przekazuje dalej nowe informacje do GUS i ZUS. W urzędzie skarbowym możemy również dokonać rejestracji na potrzeby VAT w zakresie podatku od towarów i usług, aby to zrobić wypełniamy formularz VAT-R. Dzięki temu będziemy uprawnieni do odliczania naliczonego podatku VAT.

\subsection{Uzupełnienie zgłoszenia w ZUS}


Również ZUS wymaga od nas uzupełnienia danych przekazanych do KRS w celu wpisu do Centralnego Rejestru Płatników Składek. To dzieje się po złożeniu druku NIP-8 w urzędzie skarbowym.




	
	\section{Formy opodatkowania przedsiębiorstwa}
	
	\section{Zatrudnianie pracowników}
	
	\section{Wnioski}
    
\end{document}